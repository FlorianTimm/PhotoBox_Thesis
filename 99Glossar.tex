% Photogrammetrie

\newglossaryentry{Bildweite}
{
    name=Bildweite,
    description={Abstand zwischen der Bildebene und dem Hauptpunkt der Linse, entspricht der \Gls{Kamerakonstante}, bei Fokussierung auf unendlich ist die Bildweite gleich der \Gls{Brennweite}}
}

\newglossaryentry{Brennweite}
{
    name=Brennweite,
    description={Abstand zwischen dem Hauptpunkt der Linse und dem Brennpunkt}
}

\newglossaryentry{Kamerakonstante}
{
    name=Kamerakonstante,
    description={auch Kammerkonstante, photogrammetrische Begriff für die Bildweite}
}

\newglossaryentry{Bildhauptpunkt}
{
    name=Bildhauptpunkt,
    description={Punkt, an dem die optische Achse die Bildebene schneidet}
}

\newglossaryentry{Verzeichnung}
{
    name=Verzeichnung,
    description={Verzeichnung ist die Abweichung der Abbildung eines Objektes von der idealen Abbildung. Es gibt verschiedene Arten von Verzeichnungen, siehe \autoref{s:innereorientierung}
        }
}

\newglossaryentry{aeussereOrientierung}
{
    name=äußere Orientierung,
    description={Lage und Ausrichtung der Kamera im Raum, siehe \autoref{s:aeussereorientierung}}
}

\newglossaryentry{innereOrientierung}
{
    name=innere Orientierung,
    description={Kalibrierung der Kamera, beinhaltet die \Gls{Kamerakonstante}, Lage des \Gls{Bildhauptpunkt} und der Verzeichnungskoeffizienten, siehe \autoref{s:innereorientierung}}
}

\newglossaryentry{Homografie}
{
    name=Homografie,
    description={projektive Abbildung einer Ebene auf eine andere, z.B. in zwei Bildern, siehe \autoref{s:homografie}}
}

% IT

\newglossaryentry{WLAN}
{
    name=WLAN,
    description={Wireless Local Area Network, drahtloses lokales Netzwerk}
}

\newglossaryentry{WPA2}
{
    name=WPA2,
    description={Wi-Fi Protected Access 2, Verschlüsselungsstandard für \Gls{WLAN}}
}

\newglossaryentry{Socket}
{
    name=Socket,
    description={Kommunikationsendpunkt, ermöglicht die Kommunikation zwischen zwei Prozessen}
}



% Abkürzungen

\newacronym{dpt}{dpt}{Dioptrien}

\newacronym{SfM}{SfM}{Structure-from-Motion}

\newacronym{CCCT}{CCCT}{concentric circular coded target}

\newacronym[
    description={Scale-Invariant Feature Transform, Algorithmus zur Bestimmung von Merkmalen in Bildern, die invariant gegenüber Skalierung, Rotation und Beleuchtung sind, siehe \autoref{ss:sift}}
]{SIFT}{SIFT}{Scale-Invariant Feature Transform}


\newacronym{API}{API}{Application Programming Interface}

\newacronym{HTTP}{HTTP}{Hypertext Transfer Protocol}

\newacronym[
    description={Representational State Transfer, Architekturstil für verteilte Systeme, oft in Verbindung mit Webanwendungen verwendet, basiert auf dem \gls{HTTP}-Protokoll, Zustand wird mit übertragen, so dass dieser nicht gespeichert werden muss}]{REST}{REST}{Representational-State-Transfer}

\newacronym{px}{px}{Pixel}


% Unverwendet

\newacronym{JSON}{JSON}{JavaScript Object Notation}

\newacronym{CSV}{CSV}{Comma Separated Values}

\newacronym{XML}{XML}{Extensible Markup Language}

\newacronym{HTML}{HTML}{Hypertext Markup Language}

\newacronym{URL}{URL}{Uniform Resource Locator}