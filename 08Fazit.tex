\documentclass[./00PhotoBox.tex]{subfiles}
\graphicspath{{\subfix{./img/}}}
\begin{document}



\chapter{Fazit und Ausblick}
\label{c:fazitausblick}

Es hat  sich gezeigt, dass es möglich ist ein photogrammetrisches Messsystem auf Basis von Raspberry-Pi-Kameras zu entwickeln. Die Kameras sind in der Lage, Bilder in hoher Qualität aufzunehmen und diese an eine zentrale Steuereinheit zu übertragen. Die Steuereinheit kann die Kameras synchronisieren und die Bilder an eine SfM-Software übertragen. Durch die 24 Kameras ist eine schnelle Erfassung ohne das zum Beispiel ein Drehteller notwendig wird möglich.

Die Verwendung von WLAN zur Datenübertragung hat sich als nicht optimal herausgestellt. Die Übertragungsgeschwindigkeit ist bei dieser Anzahl an Teilnehmern im Netzwerk gering, so dass die Übertragung der Bilder lange dauert. Eine Übertragung über LAN wäre hier effektiver, jedoch wäre der Verkabelungsaufwand deutlich höher und die Mobilität des Systems eingeschränkt. Da die Raspberry Pi Zero W über keinen RJ45-Anschluss verfügt, müsste ein zusätzlicher Adapter verwendet werden.

Die Verwendung des Raspberry Pi 4 als Schnittstelle vereinfacht zwar die Bedienung und ermöglicht die Verwendung des Systems ohne externe Hardware, jedoch ist die Leistung des Raspberry Pi 4 für die Verarbeitung der Bilder zum 3D-Modell nicht ausreichend. Bei der Verwendung eines externen Rechners könnte hier eine Desktop-Software alle Aufgaben übernehmen, die dieser ausführt. Neben der Kostenersparnis würde dieses auch die räumliche Skalierbarkeit des Systems erhöhen und die Komplexität verringern.


\todo{Textform}
\begin{itemize}
    \item Integration von Fokusstacking
    \item LED-Beamer mit Musterprojektion
    \item Modul 3 Wide statt "Normal"
    \item 3D-Modell in Weboberfläche zur visuellen Kontrolle der berechneten Kamera-Positionen
\end{itemize}

\biblio
\end{document}