\documentclass[./00PhotoBox.tex]{subfiles}
\graphicspath{{\subfix{./img/}}}
\begin{document}



\chapter{Fazit und Ausblick}

Die Arbeit hat gezeigt, dass es möglich ist ein photogrammetrisches Messsystem auf Basis von Raspberry-Pi-Kameras zu entwickeln. Die Kameras sind in der Lage, Bilder in hoher Qualität aufzunehmen und diese an eine zentrale Steuereinheit zu übertragen. Die Steuereinheit kann die Kameras synchronisieren und die Bilder an eine \gls{SfM}-Software übertragen.

\section{Fazit}
\todo{Textform}
\begin{itemize}
    \item LAN statt WLAN (bessere Übertragungsgeschwindigkeit)
    \item Notwendigkeit Raspberry Pi 4 (Könnte durch Java-Software ersetzt werden)
\end{itemize}

\section{Ausblick}
\label{s:ausblick}
\todo{Textform}
\begin{itemize}
    \item Integration von Fokusstacking
    \item LED-Beamer mit Musterprojektion
    \item Modul 3 Wide statt "Normal"
    \item 3D-Modell in Weboberfläche zur visuellen Kontrolle der berechneten Kamera-Positionen
\end{itemize}

\biblio
\end{document}