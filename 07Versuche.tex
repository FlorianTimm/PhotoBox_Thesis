\documentclass[./00PhotoBox.tex]{subfiles}
\graphicspath{{\subfix{./img/}}}
\begin{document}

\chapter{Untersuchungen zur Genauigkeit und Systemaufbau}
Nach Abschluss der Konstruktion und des Aufbaus des Prototyps, wurden verschiedene Untersuchungen durchgeführt, um die Genauigkeit des Systems zu überprüfen und die Anzahl der Kameras zu evaluieren. Hierzu wurden unter anderem Vergleichsmessungen mit verschiedenen Systemen durchgeführt.

\section{Genauigkeitsüberprüfung des 3D-Modelles}
Um die Genauigkeit der 3D-Modell-Erzeugung zu überprüfen, wurden mehrere Prüfkörper mit dem Prototypen und mit einem kommerziellen Streifenprojektionssystem vermessen. Die Ergebnisse wurden miteinander verglichen.

\subsection{Erwartete Genauigkeit}
Die erwartete Genauigkeit kann auf verschiedenen Wegen berechnet werden. Grundlage ist meistens der Bildmaßstab. Dieser unterscheidet sich je nach Entfernung. Die Entfernung zum Objekt beträgt im Normalfall zwischen 10 und 50~cm. Der Bildmaßstab $m$ berechnet sich nach \cite[S. 171]{luhmann} wie folgt:

\begin{align}
    m       & = \frac{h}{c}                                  \\
    m_{min} & = \frac{100~\text{mm}}{4,7~\text{mm}} = 21,28  \\
    m_{max} & = \frac{500~\text{mm}}{4,7~\text{mm}} = 106,38
\end{align}

Für die Bildmessgenauigkeit werden verschiedene Werte in der Literatur erwähnt, sie liegen je nach Messmethode zwischen 0,05 und 3~px, je nach Messmethode. Für Messung von CCCTs wird z.B. von \cite{soot2015} eine Genauigkeit von 0,1~px angegeben.
Eigene, manuelle Messungen ergaben eine Genauigkeit von etwa 1,5~px. Vereinfacht wurde mit 1~px Genauigkeit gerechnet.

\begin{align}
    dx'      & = 1~\text{px} = 0,0014~\frac{\text{mm}}{\text{px}} \cdot 12~\text{px} = 0,0014~\text{mm} \\
    dX       & = m \cdot dx'                                                                            \\
    dX_{min} & = 21,28 \cdot 0,0014~\text{mm} = 0,03~\text{mm}                                          \\
    dX_{max} & = 106,38 \cdot 0,0014~\text{mm} = 0,15~\text{mm}
\end{align}

An diesem Wert muss dann noch der sogenannte Design-Faktor angebracht werden, dieser liegt bei Rundumverbänden zwischen 0,4 und 0,8, bei Stereoaufnahmen bei 1,5-3,0 \citep[S. 174]{luhmann}. Vereinfacht für eine grobe Abschätzung wird hier daher der Faktor weggelassen. Die erwartete Genauigkeit liegt damit deutlich unter einem Millimeter in der Lage.

\begin{align}
    s_{px; min}' & = dX_{min}
    s_{px; max}' & = dX_{max}
\end{align}

Die Genauigkeit der Tiefeninformationen ist von dem Verhältnis des Abstandes der Kameras zur Entfernung abhängig. Zwischen zwei Kamerareihen sind etwa 25~cm Abstand. Die Genauigkeit der Tiefeninformationen kann daher wie folgt berechnet werden \citep[S. 174]{luhmann}:

\begin{align}
    s_Z       & = m \cdot \frac{h}{b} \cdots s_{px}'                                                   \\
    s_{Z;min} & = 21,28 \cdot \frac{10~\text{cm}}{30~\text{cm}}\cdot 0,03~\text{mm}  = 0,02~\text{mm}  \\
    s_{Z;max} & = 106,38 \cdot \frac{50~\text{cm}}{30~\text{cm}} \cdot 0,15~\text{mm} = 0,25~\text{mm}
\end{align}

Die erwartete Genauigkeit und auch die Auflösung der Kameras liegt also im Bereich eines Fünftel Millimeters.


\subsection{Vergleichsmessung mit Streifenprojektionssystem}
Als Vergleichsmessung wurden die Objekte mit einem Zeiss GOM ATOS 5 aufgemessen. Hierbei handelt es sich um ein Streifenprojektionssystem, welches hauptsächlich in der Industrie zur Vermessung von Bauteilen eingesetzt wird. Streifenprojektionssysteme arbeiten auch photogrammetrisch, haben aber den Vorteil, das sie durch das Projektionssystem auch texturarme Objekte erfassen können. Wie der Name bereits andeutet, wird ein Streifenmuster auf das Objekt projiziert, welches dann von mindestens einer Kamera aufgenommen wird. Projektor und Kamera sind auf einer festen Basis montiert. \citep[S. 581f]{luhmann}

Bei dem verwendeten System werden zwei Kameras eingesetzt, die sich ein massives Gehäuse mit dem mittig angeordneten Projektor teilen. Bei dem verwendeten Messvolumen und Objektentfernung beträgt die Genauigkeit etwa \todo{Wert}. Aufgrund der deutlich höheren erwarteten Genauigkeit des Streifenprojektionssystems, kann davon ausgegangen werden, dass die Messungen deutlich genauer sind als die des Prototypen und als wahre Werte angenommen werden können.

\subsection{Ergebnisse}

\begin{figure}
    \centering
    \includegraphics[width=0.8\textwidth]{img/moai_fehler.jpg}
    \caption{Differenzbild des Moai (rot: Fehler größer 1 mm)}
    \label{fig:differenz_moia}
\end{figure}


\section{Nutzung eines Drehtellers}


\section{Evaluation der Kameraanzahl}

\biblio
\end{document}