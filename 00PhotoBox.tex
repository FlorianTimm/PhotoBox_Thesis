\documentclass[a4paper,12pt,bibliography=totoc, listof=totoc, titlepage]{scrreprt}
\usepackage[ngerman]{babel}
\usepackage[utf8]{inputenc}
\usepackage[left=3cm,right=2.5cm,top=2.5cm,bottom=2.5cm]{geometry}
\usepackage[onehalfspacing]{setspace}
\renewcommand{\arraystretch}{1.5}
\usepackage{graphicx}
\graphicspath{{img/}}
\usepackage{color}
\usepackage[dvipsnames,dvipsnames,table,xcdraw]{xcolor}
\usepackage[toc,page]{appendix}
\usepackage[printonlyused]{acronym}
%\usepackage[scaled]{berasans}
%\renewcommand*\familydefault{\sfdefault}  %% Only if the base font of the document is to be sans serif
%\usepackage[T1]{fontenc}

% Absatz-Einstellungen
\setlength{\parindent}{15pt} % Einrücken
\setlength{\parskip}{6pt}  % Horizontaler Abstand

\usepackage{subcaption}
\usepackage{graphicx} % omit 'demo' for real document
  
\usepackage[hyphens]{url}
\usepackage[hidelinks]{hyperref}
\setlength {\marginparwidth }{2cm}
\usepackage{todonotes}
\usepackage{amsmath}
\usepackage{cancel}
\usepackage{scrhack}

% Kommentare mit \begin{comment}
\usepackage{verbatim}

\usepackage{multirow}

\newcommand*\justify{%
  \fontdimen2\font=0.4em% interword space
  \fontdimen3\font=0.2em% interword stretch
  \fontdimen4\font=0.1em% interword shrink
  \fontdimen7\font=0.1em% extra space
  \hyphenchar\font=`\-% allowing hyphenation
}

\newcommand{\code}[1]{\texttt{\justify{#1}}}
%\usepackage{tocloft}

%Boxfehler
\hbadness=1000000

% Hurenkinder und Schusterjungen verhindern
\clubpenalties 4 10000 10000 1000 1000
\widowpenalties 4 10000 10000 1000 1000
\displaywidowpenalty=10000

% Listings
\usepackage{listings}
\lstset{
   breaklines=true,
   captionpos=t,
   basicstyle=\scriptsize\ttfamily,
   keywordstyle=\bfseries\ttfamily\color{orange},
   stringstyle=\color{green}\ttfamily,
   commentstyle=\color{gray}\ttfamily,
   emph={square}, 
   emphstyle=\color{blue}\texttt,
   emph={[2]root,base},
   emphstyle={[2]\color{yac}\texttt},
   showstringspaces=false,
   flexiblecolumns=false,
   tabsize=2,
   numbers=left,
   numberstyle=\tiny,
   numberblanklines=false,
   stepnumber=1,
   numbersep=10pt,
   xleftmargin=15pt
 }

% Zitierstil
%\usepackage[style=authoryear,citestyle=authoryear,natbib=true]{biblatex}
%\bibliography{Thesis.bib}
\usepackage[round]{natbib}
\bibliographystyle{hcu}
\def\biblio{\bibliography{00PhotoBox} \printglossaries}


\usepackage[acronym, toc]{glossaries}

\makeglossaries

\loadglsentries{99Glossar.tex}


\usepackage{subfiles}

\title{Aufbau eines photogrammetrischen Messsystems unter Verwendung von Raspberry-Pi-Kameras als Low-Cost-Sensoren}
\author{Florian Timm}
\date{August 2024}


\begin{document}
\def\biblio{}
\pagenumbering{Roman}
\begin{titlepage}
    \begin{center}
        \renewcommand{\arraystretch}{0.7}
        \begin{tabular}{lr}
            \begin{tabular}{l}
                \includegraphics[width=0.35\textwidth]{img/hculogo_grau.png}
            \end{tabular} \hspace{1.2cm} &
            \begin{tabular}{r}
                Universität für           \\Baukunst und Metropolenentwicklung\\
                Henning-Voscherau-Platz 1 \\
                20457 Hamburg             \\
            \end{tabular}
        \end{tabular}\\\vspace{5cm}
        \doublespacing
        {\huge\bfseries Aufbau eines photogrammetrischen Messsystems unter Verwendung von Raspberry-Pi-Kameras als Low-Cost-Sensoren }\vspace{0.5cm}\\

        {\large\bfseries Geodäsie und Geoinformatik\\Masterthesis\\Sommersemester 2024}\vspace{2cm}\\
        {\large Florian Timm}

        % hspace und vspace bedeutet horizontaler bzw. vertikaler Abstand
        \vspace{7cm}
        Abgabedatum: 01. August 2024
    \end{center}
    \setcounter{page}{0} % Seitenzahl wird auf 0 gesetzt 
\end{titlepage}


\vspace{2cm}
\noindent\textbf{\large Verfasser}\\
Florian Timm\\
Matrikelnummer: 6028121\\
Gaiserstraße 2, 21073 Hamburg\\
\\
E-Mail: florian.timm@hcu-hamburg.de\\
\vspace{3cm}\\
\noindent\textbf{\large Erstprüfer}\\
Prof. Dr.-Ing. Thomas Kersten\\
HafenCity Universität Hamburg\\
Überseeallee 16, 20457 Hamburg\\
\\
E-Mail: thomas.kersten@hcu-hamburg.de\\
\vspace{3cm}\\
\textbf{\large Zweitprüfer}\\
Dipl.-Ing. Kay Zobel\\
HafenCity Universität Hamburg\\
Überseeallee 16, 20457 Hamburg\\
\\
E-Mail: kay.zobel@hcu-hamburg.de\\
\newpage
\noindent\textbf{\large Kurzzusammenfassung}\\
Die Photogrammetrie bietet die Möglichkeit, mit relativ einfacher Technik 3D-Modelle zu erstellen. Die Aufnahme der Bilder ist jedoch sehr zeitaufwendig und daher für die Erfassung vieler Objekte, z.B. bei der Digitalisierung von Museumsobjekten, nicht praktikabel. Systeme mit mehreren fest installierten Kameras setzen in der Regel auf hochwertige Kameras, die jedoch die Hardwarekosten stark in die Höhe treiben.

In dieser Arbeit soll der Lösungsansatz untersucht werden, mehrere kostengünstige Kameras, die fest auf einem Rahmen montiert sind, zu verwenden. Mit Kameras für den Raspberry Pi soll ein photogrammetrisches Messsystem für kleine Objekte aufgebaut werden. Dazu soll eine Schnittstelle zur Synchronisation der Kameras programmiert und eine Möglichkeit zur Kalibrierung der Kameras entwickelt werden. Das Endergebnis soll es im Idealfall auch einem photogrammetrischen Laien ermöglichen, schnell und ohne lange Einarbeitungszeit 3D-Modelle in akzeptabler Auflösung und Qualität zu erzeugen.
\vspace{2cm}\\
\noindent\textbf{\large Abstract}\\
Photogrammetry offers the possibility of creating 3D models with relatively simple technology. However, capturing the images is very time consuming and therefore not practical for many objects, such as museum digitisation. Systems using multiple fixed cameras usually rely on high quality cameras, which significantly increases hardware costs.

This paper explores the solution of using multiple low-cost cameras mounted on a frame. Cameras for the Raspberry Pi are used to build a photogrammetric measurement system for small objects. This involves programming an interface to synchronise the cameras and developing a way to calibrate the cameras. The end result should ideally enable even a photogrammetric layman to produce 3D models of acceptable resolution and quality quickly and without a long training period.

% Mehrere gleichzeitig zitieren
\providecommand{\citeTwo}[4]{\citep[{\citealp[#1]{#2};}][#3]{#4}}
\providecommand{\citeThree}[6]{\citep[{\citealp[#1]{#2}; \citealp[#3]{#4};}][#5]{#6}}
\providecommand{\citeFour}[8]{\citep[{\citealp[#1]{#2}; \citealp[#3]{#4}; \citealp[#5]{#6};}][#7]{#8}}

\newpage

\tableofcontents
\newpage

\pagenumbering{arabic}
\setcounter{page}{1}

\subfile{01Einleitung}

\subfile{02Grundlagen}

\subfile{03Aufbau}

\subfile{04Voruntersuchungen}

\subfile{05Software}

\subfile{06Kalibrierung}

\subfile{07Versuche}

\subfile{08Fazit}

\clearpage

\printglossaries

\clearpage
%Literatur
\renewcommand\UrlFont\itshape
\renewcommand{\refname}{Literaturverzeichnis}
\bibliography{00PhotoBox}
%\printbibliography
\listoffigures
\listoftables
%\lstlistoflistings


\renewcommand{\appendixpagename}{\appendixname}
\renewcommand{\appendixtocname}{\appendixname}
\begin{appendices}
    \subfile{A01Bedienung}
    \subfile{A02Teile}
\end{appendices}

\newpage
\thispagestyle{empty}
\noindent\textbf{\large Erklärung}\\
Hiermit versichere ich, dass ich die beiliegende Master-Thesis ohne fremde Hilfe selbst\-stän\-dig verfasst und nur die angegebenen Quellen und Hilfsmittel benutzt habe.\\
\\
Wörtlich oder dem Sinn nach aus anderen Werken entnommene Stellen sind unter Angabe der Quellen kenntlich gemacht.
\\
\\
\\
\\
\noindent{\underline{Hamburg, den 01. Sept. 2024~\hspace{9cm}}}\\
\noindent{\small Ort, Datum \hspace{4.5cm} Florian Timm}


\end{document}