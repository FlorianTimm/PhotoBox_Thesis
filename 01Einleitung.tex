\documentclass[./00PhotoBox]{subfiles}
\graphicspath{{\subfix{./img/}}}
\begin{document}

\chapter{Einleitung}
\label{c:einleitung}

Im Rahmen dieser Arbeit soll ein Photogrammetrie-System auf Basis von Raspberry-Pi-Kameras entwickelt werden. Das System soll es ermöglichen, kleine Objekte mit geringem finanziellen und personellen Aufwand zu erfassen und daraus 3D-Modelle zu erstellen.

Die folgenden Abschnitte geben einen Überblick über die Motivation und das Konzept der Arbeit und stellen den bisherigen Stand der Technik dar.

\section{Motivation}

In Museen besteht vielfach der Wunsch, den Bestand an Exponaten zu digitalisieren, beispielsweise um diese online in virtuellen Ausstellungen zu präsentieren. Entsprechende Handreichungen des Deutschen Museumsbundes legen auch die Digitalisierung als 3D-Modelle nahe, verweisen aber auf den großen Aufwand und Format-Probleme \citep[S. 43]{handreichung_digital}. Neben der reinen Präsentation der Bestände ist aber auch die Dokumentation und Erforschung der Exponate von Interesse. Beispielsweise durch Naturkatastrophen, Brände und bewaffnete Konflikte können Kulturgüter jederzeit verloren gehen oder beschädigt werden. Die Digitalisierung ermöglicht es, diese Objekte zu bewahren und der Öffentlichkeit zugänglich zu machen. Neben der vorsorglichen Digitalisierung ist aber auch die schnelle Erfassung von Objekten nach einem Schadensfall wichtig \citep{kulturgutretter}.

Auch in vielen weiteren Bereichen besteht der Bedarf dreidimensionale Modelle einfach und kostengünstig zu erfassen, beispielsweise in der Archäologie, der Spiele- und Filmindustrie für die 3D-Modellierung oder auch der Industrie zur Entwicklung und Qualitätskontrolle \citep[S. 37f]{luhmann}.

\section{Konzept}
Das zu entwickelnde System soll dabei auch von Laien mit kurzer Einarbeitungszeit bedienbar sein. Dazu ist eine weitgehende Automatisierung der Schritte erforderlich. Auch ein eigener Nachbau des Systems soll einfach möglich sein. Um Lizenzkosten zu sparen, ist die Möglich\-keit der Nutzung von Open-Source-Software zu prüfen. Das System soll kleine Objekte in einer Größenordnung bis etwa \SI{40}{\centi\metre} Durchmesser erfassen können.

Nach einer Analyse der Anforderungen an ein solches System ist die Entwicklung eines Prototyps vorgesehen, welcher aus mehreren Kameras besteht, die auf einem Rahmen montiert sind. Die Anordnung der Kameras erfolgt dabei so, dass eine Erfassung des Objekts aus verschiedenen Blickwinkeln gewährleistet ist. Die Kameras sind dabei synchron auszulösen und die Daten anschließend automatisch zu übertragen, sodass eine direkte Verarbeitung der Aufnahmen zu einem 3D-Modell erfolgen kann.

Zur Untersuchung möglicher Optimierungen des Systems ist neben der eigentlichen Entwicklung und Genauigkeitsuntersuchung eine Analyse der Anzahl der Kameras auch unter Nutzung eines Drehtellers vorgesehen. Ziel ist die Identifikation von Potenzialen zur Senkung der Hardwarekosten oder alternativ zur Steigerung der Auflösung und Genauigkeit.

Zusammengefasst sollen folgende Anforderungen erfüllt werden:

\begin{enumerate}
    \item Erfassung von kleinen Objekten bis etwa \SI{40}{\centi\metre} Durchmesser
    \item automatisierte Erfassung der Bilder
    \item automatisierte Übertragung der Bilder
    \item automatisierte Verarbeitung der Bilder zu einem 3D-Modell
    \item einfache Bedienbarkeit
    \item geringe Kosten
    \item einfacher Nachbau
    \item Nutzung von Open-Source-Software
    \item Transportmöglichkeit/Nutzung in anderen Ländern
    \item Möglichkeit zur Erweiterung
\end{enumerate}

\section{Stand der Technik}
Der Ansatz, Kameras auf einem festen Rahmen zu montieren, Bilder aufzunehmen und anschließend automatisiert 3D-Modelle auf Basis von Photogrammetrie zu erzeugen, ist bereits weit verbreitet und erprobt. Hauptsächlich unterscheiden sich diese im Vergleich zu dem hier untersuchten Ansatz in der Wahl der Kameras und der Zielgruppe der Bediener. Häufig werden hochwertige Kameras verwendet, die jedoch entsprechend hohe Kosten in der Anschaffung verursachen.

Bereits Anfang der 1990er Jahre wurde ein modulares System von Leica entwickelt, das die Objekterfassung mittels mehrerer Digitalkameras ermöglichte, das sogenannte Leica POM. Dieses System wurde für industrielle Anwendungen entwickelt, beispielsweise für die Qualitätskontrolle von Bauteilen. Es ermöglichte die flexible Nutzung verschiedener Kamerasysteme, Drehteller und Lichtquellen. Ähnlich wie bei dem hier geplanten System wurden die Daten auch automatisch übertragen und verarbeitet. Auch konnten bereits einige Punkte automatisch gemessen werden. Ähnlich wie das aktuelle System TubeInspect \citep{aicon_tubeinspect} von Hexagon - dem Mutterkonzern von Leica und AICON - nutzt es jedoch kein \Gls{SfM} zur Erstellung der 3D-Modelle, sondern Kantendetektion und anschließenden Bildvergleich. Hierfür wird das Objekt von hinten beleuchtet und die Kanten erfasst. Die Genauigkeit des Systems wird mit \SI{0,1}{\milli\metre} angegeben.
\citep{leica_pom_concept}

Auch Raspberry-Pi-Kameras wurden bereits in der Photogrammetrie eingesetzt, beispielsweise bei dem Projekt Pi3DScanner. Dieses verfolgt das Ziel, ein kostengünstiges Photo\-grammetrie-System auf Basis mehrerer Raspberry Pi zu entwickeln. Der Ansatz ist ähnlich, jedoch kommerziell und aufgrund der Nutzung einer größeren Anzahl von Kameras höherpreisiger als das hier geplante Vorhaben. Zudem liegt der Fokus, wie bei den meisten Projekten dieser Art, auf der Erfassung von größeren Objekten oder Personen.
\citep{pi3dscanner}

An der HafenCity Universität wurde 2015 ein Innenraum-Erfassungssystem entwickelt, das auf Basis einer Raspberry-Pi-Kamera und einem Laserentfernungsmessgerät arbeitet. Hierbei wurde auch die Genauigkeit der Kameras untersucht und die Möglichkeit der Kalibrierung geprüft. Bei dem verwendeten, älteren Raspberry Pi Camera Module v2 handelt es sich jedoch um eine Kamera mit Fixfokus, wodurch die Kalibrierung vereinfacht ist (siehe \autoref{s:kameras}). Hier wurde dennoch ein instabiler \Gls{Bildhauptpunkt} festgestellt, dass Genauigkeitspotenzial nach Simultankalibrierung aber mit einer Spiegelreflexkamera vergleichbar bezeichnet.
\citep{3d_raspi_laserscanner}

Ansätze für die Erfassung von kleineren Objekten in Kombination von Raspberry-Pi-Kamera-Modulen sind bisher nicht bekannt. Problematisch ist hierbei die geringe Tiefenschärfe im Makrobereich (vgl. \autoref{s:schaerfe}). Dieses Problem tritt aber auch bei höherwertigen Kameras und noch kleineren Objekten auf, bei denen ein Abblenden des Objektives nicht mehr für ausreichend Schärfentiefe sorgt. Auch im Bereich der Mikroskopie ist dieses Problem weit verbreitet. \cite{focusstack_sfm} zeigt hier eine Lösung durch die Nutzung von Fokusstacking, bei der mehrere Bilder mit unterschiedlichen Fokussierungen aufgenommen und anschließend zusammengefügt werden. Dieser Ansatz wurde auch in dieser Arbeit geprüft.

Der Einsatz von \acrfull{SfM} -- die automatische Erstellung von 3D-Modellen aus Bildern (siehe \autoref{c:photogrammetrie}) -- wurde für die Erfassung von Kulturgütern bereits \citeyear{kersten2012} an der HafenCity untersucht \citep{kersten2012}. Hierbei wurden zum Teil mit Laserscannern vergleichbare Ergebnisse erzielt, wobei Objekte verschiedener Größenordnungen wie Gebäude oder kleine Objekte überprüft wurden. Manche Testobjekte, vorallem mit texturarmen Oberflächen, führten jedoch zu unbrauchbaren Ergebnissen, aber eine stetige Verbesserung der Software war schon damals absehbar. Auch die Nutzung von Open-Source-Software wurde bereits untersucht. Genauigkeitsuntersuchungen von \cite{IvanNikolov} an der Universität Aalborg zeigten schon bessere Ergebnisse, jedoch war auch hier je nach Software bei einigen Objekten keine Erzeugung des 3D-Modelles möglich, beispielsweise bei glänzenden oder texturlosen Oberflächen.

In der praktischen Arbeit zur Dokumentation von Kulturgütern ist \acrfull{SfM} bereits angekommen. Bei archäologischen Grabungen ist die Nutzung zur Dokumentation der Grabungsflächen inzwischen üblich \citep{grabungen_sfm}. Hier hat es sich \Gls{SfM} als Ergänzung zur Tachymetrie und als Alternative zum Laserscanning durchgesetzt.

Auch zur Rettung von mobilen Kulturgütern wird die Nutzung geprüft. Das Projekt KulturGutRetter des Deutschen Archäologischen Instituts und des Technischen Hilfswerkes hat sich zum Ziel gesetzt, durch verschiedene Gefahren wie Naturkatastrophen und Kriegen gefährdete Kulturgüter zu dokumentieren und zu möglichst zu retten. Hierbei wird auch \Gls{SfM} als mögliche Technik zur Dokumentation von mobilen aber auch immobilen Kulturgütern genannt \citep[S. 48]{kgr_article}. Hierzu wird ein mobiles, flugzeugverlastbares Rettungslabor genutzt. Für die grundlegende Dokumentation von mobilen Kulturgütern wird hier bisher eine Photobox mit einer Kamera verwendet. Die Systeme sind noch in der Weiterentwicklung, sodass eine Integration von \Gls{SfM} denkbar ist. Die Bedienung ist hierbei durch Fachpersonal vorgesehen. \citep{kulturgutretter}

\biblio
\end{document}