\documentclass[./00PhotoBox]{subfiles}
\graphicspath{{\subfix{./img/}}}
\begin{document}

\chapter{Einleitung}

Im Rahmen dieser Arbeit soll ein Photogrammetrie-System auf Basis von Raspberry-Pi-Kameras entwickelt werden. Das System soll es ermöglichen, kleine Objekte mit geringem personellen Aufwand zu erfassen und daraus 3D-Modelle zu erstellen.

Die folgenden Abschnitte geben einen Überblick über die Motivation und das Konzept der Arbeit und stellen den Stand der Thematik dar.

\section{Motivation}

In Museen besteht vielfach der Wunsch, ihre Exponate zu digitalisieren. Entsprechende Handreichungen des Deutschen Museumsbundes legen auch die Digitalisierung als 3D-Modelle nahe, verweisen aber auf den großen Aufwand und Format-Probleme \citep[S. 43]{handreichung_digital}.
Auch in vielen weiteren Bereichen besteht der Bedarf dreidimensionale Modelle einfach und kostengünstig zu erfassen, beispielsweise in der Archäologie, der Spiele- und Filmindustrie oder der Industrie. \todo{Quellen}

Das System soll dabei auch von Laien mit kurzer Einarbeitungszeit bedienbar sein. Die erforderlichen Schritte sollen dafür möglichst automatisiert werden. Auch der Nachbau des Systems soll einfach möglich sein. Um Lizenzkosten zu sparen, soll die Möglich\-keit Open-Source-Software zu nutzen geprüft werden. Als Größenordnung soll das System kleine Objekte bis etwa 40~cm Durchmesser erfassen können.

\section{Konzept}
Nach einer Recherche zu den Anforderungen an ein solches System soll ein Prototyp entwickelt werden. Dieser soll aus mehreren Kameras bestehen, die auf einem Rahmen montiert sind. Die Kameras sollen dabei so angeordnet sein, dass sie das Objekt aus verschiedenen Blickwinkeln erfassen können und sollen dabei über eine Schnittstelle synchronisiert werden. Die Datenübertragung soll danach automatisiert erfolgen und die Bilder sollen automatisch zu einem 3D-Modell verarbeitet werden.

Neben der eigentlichen Entwicklung und Untersuchung dieses Systemes soll abschließend die Anzahl der Kameras und die Nutzung eines Drehtellers evaluiert werden, um hiermit gegebenenfalls die Hardwarekosten weiter zu senken oder die Auflösung und Genauigkeit zu steigern.

Zusammengefasst sollen folgende Anforderungen erfüllt werden:

\begin{enumerate}
    \item Erfassung von kleinen Objekten bis etwa 40~cm Durchmesser
    \item Automatisierte Erfassung der Bilder
    \item Automatisierte Übertragung der Bilder
    \item Automatisierte Verarbeitung der Bilder zu einem 3D-Modell
    \item Einfache Bedienbarkeit
    \item Geringe Kosten
    \item Einfacher Nachbau
    \item Nutzung von Open-Source-Software
    \item Transportmöglichkeit/Nutzung in anderen Ländern
    \item Möglichkeit zur Erweiterung
\end{enumerate}

\section{Stand der Technik}
Der Ansatz, Kameras auf einem festen Rahmen zu montieren, Bilder aufzunehmen und anschließend automatisiert 3D-Modelle auf Basis von Photogrammetrie zu erzeugen, ist bereits weit verbreitet und erprobt. Hauptsächlich unterscheiden sich diese in der Wahl der Kameras und der Zielgruppe der Bediener. Häufig werden hochwertige Kameras verwendet, die jedoch entsprechend hohe Kosten in der Anschaffung verursachen.

Bereits Anfang der 1990er Jahre wurde ein modulares System von Leica entwickelt, das die Objekterfassung mittels mehrerer Digitalkameras ermöglichte, das sogenannte Leica POM.  Dieses System wurde für die Industrie entwickelt, beispielsweise für die Qualitätskontrolle von Bauteilen. Es ermöglichte die flexible Nutzung verschiedener Kamerasysteme, Drehteller und Lichtquellen. Ähnlich wie bei dem hier geplanten System wurden die Daten auch automatisch übertragen und verarbeitet. Auch konnten bereits einige Punkte automatisch durch Bildvergleich und Kantendetektion gemessen werden.
\citep{leica_pom_concept}

Auch Raspberry Pi Kameras wurden bereits in der Photogrammetrie eingesetzt, beispielsweise beim Pi3DScanner \citep{pi3dscanner}. Pi3DScanner ist ein Projekt, das sich auch zum Ziel gesetzt hat, ein kostengünstiges Photo\-gram\-metrie-System auf Basis von mehreren Raspberry Pi zu entwickeln. Der Ansatz ist ähnlich, die Entwicklung ist jedoch kommerziell und der Ansatz etwas höherpreisiger als das hier geplante, da deutlich mehr Kameras genutzt werden. Außerdem liegt hier, wie bei den meisten Projekten dieser Art der Fokus auf der Erfassung von größeren Objekten oder Personen.
\citep{pi3dscanner}

An der HafenCity Universität wurde 2015 ein Innenraum-Erfassungssystem entwickelt, das auf Basis einer Raspberry Pi Kamera und einem Laserentfernungsmessgerät arbeitet. Hierbei wurde auch die Genauigkeit der Kameras untersucht und die Möglichkeit der Kalibrierung geprüft. Bei dem Module 3 handelt es sich jedoch um eine Kamera mit Fixfokus, was die Kalibrierung vereinfacht. Hier wurde zwar der instabile Bildhauptpunkt bemängelt, dass Genauigkeitspotenzial nach Simultankalibrierung aber als mit einer Spiegelreflexkamera vergleichbar bezeichnet.
\citep{3d_raspi_laserscanner}

Ansätze für kleinere Objekte, bei den die Schärfentiefe der Raspberry Pi Kameras ein Problem darstellt, sind bisher nicht bekannt. Jedoch gibt es ähnliche Probleme auch bei höherwertigen Kameras und noch kleineren Objekten, bei denen ein Abblenden des Objektives nicht mehr für ausreichend Schärfentiefe sorgt. Auch tritt das Problem bei der Verwendung von Mikroskopen auf. \cite{focusstack_sfm} zeigt hier eine Lösung durch die Nutzung von Fokusstacking, bei der mehrere Bilder mit unterschiedlichen Fokussierungen aufgenommen und anschließend zusammengefügt werden. Dieser Ansatz wurde auch in dieser Arbeit geprüft.

\todo{Tube Inspect AICON}

\todo{Ivan Nikolov Uni Aarlborg}


\biblio
\end{document}