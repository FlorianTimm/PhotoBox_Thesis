\documentclass[./00PhotoBox.tex]{subfiles}
\graphicspath{{\subfix{./img/}}}
\begin{document}


\chapter{Einleitung}

\section{Konzept}

In Museen besteht vielfach der Wunsch, ihre Exponate zu digitalisieren. Entsprechende Handreichungen des Deutschen Museumsbundes legen auch die Digitalisierung als 3D-Modelle nahe, verweisen aber auf den großen Aufwand und Format-Probleme \citep[S. 43]{handreichung_digital}.
Auch in vielen weiteren Bereichen besteht der Bedarf dreidimensionale Modelle einfach und kostengünstig zu erfassen, beispielsweise in der Archäologie, der Spiele- und Filmindustrie oder der Industrie. \todo{Quellen}

Im Rahmen dieser Arbeit soll ein Kamerasystem basierend auf Raspberry-Pi-Kame\-ras entwickelt werden und untersucht werden, inwiefern es diesen Anforderungen gerecht wird. Es soll mittels Photogrammetrie mit geringen personellen Aufwand kleine Objekte bis etwa 40~cm Durchmesser erfassen. Die Bedienung soll dabei auch von Laien mit kurzer Einarbeitungszeit möglich sein und das System die meisten Schritte selbstständig durchführen. Auch der Nachbau des Systemes soll einfach möglich sein. Um Lizenzkosten zu sparen, soll die Möglichkeit Open-Source-Software zu nutzen geprüft werden.

Als Prototyp soll ein System mit 24 Kameras gebaut werden. Neben der eigentlichen Entwicklung und Untersuchung dieses Systemes soll abschließend die Anzahl der Kameras und die Nutzung eines Drehtellers evaluiert werden, um hiermit gegebenenfalls die Hardwarekosten weiter zu senken oder die Auflösung und Genauigkeit zu steigern.


\section{Stand der Thematik}
Der Ansatz, Kameras auf einem festen Rahmen zu montieren, Bilder aufzunehmen und anschließend automatisiert 3D-Modelle auf Basis von Photogrammetrie zu erzeugen, ist bereits weit verbreitet und erprobt. Hauptsächlich unterschieden sich diese in der Wahl der Kameras und der Zielgruppe der Bediener. Häufig wurden hochwertige Kameras verwendet, die jedoch hohe Kosten in der Anschaffung verursachen. Beispielsweise wurde schon 1990 ein System von Leica entwickelt,\todo{zu Ende führen...}


\paragraph{Pi3DScan}
ist ein Projekt, das sich zum Ziel gesetzt hat, ein kostengünstiges Photo\-gram\-metrie-System auf Basis von Raspberry-Pi-Kameras zu entwickeln. Der Ansatz ist ähnlich und das System bietet auch die Möglichkeit, flexibel die Anzahl der Kameras festzulegen.Die Entwicklung ist jedoch kommerziell und der Ansatz etwas höherpreisiger als das hier geplante. Außerdem ist hier der Fokus eher auf größere Objekte gelegt, die andere Ansätze erfordern.
\citep{pi3dscanner}


\biblio
\end{document}