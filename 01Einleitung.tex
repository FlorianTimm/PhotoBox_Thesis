\documentclass[./00PhotoBox.tex]{subfiles}
\graphicspath{{\subfix{./img/}}}
\begin{document}


\chapter{Einleitung}

\section{Konzept}

In Museen besteht vielfach der Wunsch, ihre Exponate zu digitalisieren. Entsprechende Handreichungen des Deutschen Museumsbundes legen auch die Digitalisierung als 3D-Modelle nahe, verweisen aber auf den großen Aufwand und Format-Probleme \citep[S. 43]{handreichung_digital}.
Auch in vielen weiteren Bereichen besteht der Bedarf dreidimensionale Modelle einfach und kostengünstig zu erfassen, beispielsweise in der Archäologie, der Spiele- und Filmindustrie oder der Industrie. \todo{Quellen}

Im Rahmen dieser Arbeit soll ein Kamerasystem basierend auf Raspberry-Pi-Kame\-ras entwickelt werden und untersucht werden, inwiefern es diesen Anforderungen gerecht wird. Es soll mittels Photogrammetrie mit geringen personellen Aufwand kleine Objekte bis etwa 40~cm Durchmesser erfassen. Die Bedienung soll dabei auch von Laien mit kurzer Einarbeitungszeit möglich sein und das System die meisten Schritte selbstständig durchführen. Auch der Nachbau des Systemes soll einfach möglich sein. Um Lizenzkosten zu sparen, soll die Möglichkeit Open-Source-Software zu nutzen geprüft werden.

Als Prototyp soll ein System mit 24 Kameras gebaut werden. Neben der eigentlichen Entwicklung und Untersuchung dieses Systemes soll abschließend die Anzahl der Kameras und die Nutzung eines Drehtellers evaluiert werden, um hiermit gegebenenfalls die Hardwarekosten weiter zu senken oder die Auflösung und Genauigkeit zu steigern.


\section{Stand der Thematik}
Der Ansatz, Kameras auf einem festen Rahmen zu montieren, Bilder aufzunehmen und anschließend automatisiert 3D-Modelle auf Basis von Photogrammetrie zu erzeugen, ist bereits weit verbreitet und erprobt. Hauptsächlich unterscheiden sich diese in der Wahl der Kameras und der Zielgruppe der Bediener. Häufig werden hochwertige Kameras verwendet, die jedoch entsprechend hohe Kosten in der Anschaffung verursachen.

Bereits Anfang der 1990er Jahre wurde ein modulares System von Leica entwickelt, das die Objekterfassung mittels mehrerer Digitalkameras ermöglichte.  Dieses System wurde für die Industrie entwickelt und ermöglichte flexible Nutzung verschiedener Kamerasysteme, Drehteller und Lichtquellen \todo{fortführen}
\citep{leica_pom}

Auch Raspberry Pi Kameras wurden bereits in der Photogrammetrie eingesetzt, beispielsweise beim Pi3DScanner \citep{pi3dscanner}. Pi3DScanner ist ein Projekt, das sich auch zum Ziel gesetzt hat, ein kostengünstiges Photo\-gram\-metrie-System auf Basis von mehreren Raspberry Pi zu entwickeln. Der Ansatz ist ähnlich, die Entwicklung ist jedoch kommerziell und der Ansatz etwas höherpreisiger als das hier geplante, da deutlich mehr Kameras genutzt werden. Außerdem liegt hier, wie bei den meisten Projekten dieser Art der Fokus auf der Erfassung von größeren Objekten oder Personen.
\citep{pi3dscanner}

Ansätze für kleinere Objekte, bei den die Schärfentiefe der Raspberry Pi Kameras ein Problem darstellt, sind bisher nicht bekannt. Jedoch gibt es ähnliche Probleme auch bei höherwertigen Kameras und noch kleineren Objekten, bei denen ein Abblenden des Objektives nicht mehr für ausreichend Schärfentiefe sorgt. Auch tritt das Problem bei der Verwendung von Mikroskopen auf. \cite{focusstack_sfm} zeigt hier eine Lösung durch die Nutzung von Fokusstacking, bei der mehrere Bilder mit unterschiedlichen Schärfepunkten aufgenommen und anschließend zusammengefügt werden. Dieser Ansatz wurde auch in dieser Arbeit geprüft.


\biblio
\end{document}