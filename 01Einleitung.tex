\documentclass[./00PhotoBox.tex]{subfiles}
\graphicspath{{\subfix{./img/}}}
\begin{document}


\chapter{Einleitung}

\section{Konzept}

In Museen besteht vielfach der Wunsch, ihre Exponate zu digitalisieren. Entsprechende Handreichungen des Deutschen Museumsbundes legen auch die Digitalisierung als 3D-Modelle nahe, verweisen aber auf Aufwand und Format-Probleme \citep[S. 43]{handreichung_digital}.
Auch bei Ausgrabungen aber auch in anderen Bereichen besteht der Bedarf dreidimensionale Modelle einfach und kostengünstig zu erfassen.

Im Rahmen dieser Arbeit soll ein Kamerasystem basiert auf Raspberry-Pi-Kameras entwickelt und untersucht werden, in wie weit es diesen Anforderungen gerecht wird. Es soll mittels Photogrammetrie mit geringen personellen Aufwand kleine Objekte bis etwa 40~cm Durchmesser erfassen. Die Bedienung soll dabei auch von Laien mit kurzer Einarbeitungszeit möglich sein und das System die meisten Schritte selbstständig durchführen. Auch der Nachbau des Systemes soll einfach möglich sein. Um Lizenzkosten zu sparen, soll die Möglichkeit OpenSource-Software zu nutzen geprüft werden.

Als Prototyp soll ein System mit 24 Kameras gebaut werden. Neben der eigentlichen Entwicklung und Untersuchung dieses Systemes soll abschließend die Anzahl der Kameras und die Nutzung eines Drehtellers evaluiert werden, um hiermit gegebenenfalls die Hardwarekosten weiter zu senken oder die Auflösung und Genauigkeit zu steigern.


\section{Stand der Thematik}

Es gab bereits einige Arbeiten zu ähnlichen Themen. Hauptsächlich unterschieden diese sich in der Wahl der Kameras und der Zielgruppe der Bediener. Häufig wurden hochwertige Kameras verwendet, die jedoch hohe Kosten in der Anschaffung verursachen. Beispielsweise wurde schon 1990 ein System von Leica entwickelt,\todo{zu Ende führen...}


\subsection{Pi3DScan}



\biblio
\end{document}