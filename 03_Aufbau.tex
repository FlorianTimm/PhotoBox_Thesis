\documentclass[./00_PhotoBox.tex]{subfiles}
\graphicspath{{\subfix{./img/}}}
\begin{document}


\chapter{Aufbau des Messsystemes}
Die Kameras sollten eine höhe geometrische Auflösung und möglichst stabile innere Orientierung aufweisen. Außerdem sollen sie während einer Messkampagne nicht in ihrer Lage zueinander verändert werden, damit die äußere Orientierung größtenteils unverändert bleibt. Daher ist ein stabiler Rahmen notwendig, an welchem die Kameras verdrehsicher angebracht werden können. Kleinere Restfehler in den Orientierungen können mit ausgeglichen werden.
Um Ungenauigkeiten durch Bewegungen zu verhindern, müssen die Kameras möglichst zeitgleich auslösen. Daher ist eine gemeinsame Steuerung und Kommunikation zwischen den Kameras notwendig. Außerdem sollen alle Bilder dann auf das Steuerungssystem übertragen werden, hierfür wir eine Form der Datenübertragung benötigt. Damit die Bilder möglichst schattenfrei ausgeleuchtet werden, muss Beleuchtung mit eingeplant werden. Außerdem muss die Stromversorgung der einzelnen Kameras sichergestellt sein.

Aus diesen Anforderungen ergeben sich die einzelnen Abschnitte dieses Kapitels.

\section{Kameras}
Als Kameras wurde das Raspberry Pi Camera Module 3 verwendet, welches jeweils von einem Raspberry Pi Zero W gesteuert wird. Im Vergleich zu anderen günstigen Kameras wie Webcams oder der ESP32 CAM haben die Kameras eine hohe geometrische Auflösung von 12 Megapixeln und dennoch mit $1,4~\mu m$ relativ große Pixel \citep{raspi_cam_datasheet}, was im subjektiven Eindruck eine sehr gute Bildqualität ergibt.

Nachteil und Vorteil zugleich ist, dass die Kamera über einen Autofokus verfügt, der aber auch elektronisch gesteuert manuell fokussieren kann. Dieser verschlechtert die Stabilität der inneren Orientierung (vgl. \autoref{s:innereorientierung}) weiter und wurde daher auch besonders im \todo{Verweis zu Fokusexperimenten} analysiert. Da die Bilder aber Nahbereich zwischen $20$ und $70~\text{cm}$ benötigt werden, ist hier die Schärfentiefe niedrig. Der elektronische Fokus, eine wiederholgenaue und damit mathematisch modellierbare Fokussierung vorausgesetzt, ermöglicht hier, Fokusstacking zu nutzen um den Schärfebereich zu vergrößern.

Weiterer Vorteil der Lösung mit einzelnen Raspberry Pis ist es, dass hierdurch bereits die einzelnen Kameraeinheiten Berechnungen wie das Identifizieren von Passpunkten übernehmen könnten und auch durch die Nutzung von Netzwerkverbindungen für die Steuerung das System skalierbar im Sinne der Anzahl der Kameras aber auch der Größenordnung der Abstände zwischen den Kameras.

\section{Rahmen}
Der Rahmen muss möglichst stabil sein, damit die Kameras sich nicht in ihrer Lage verändern können. Jedoch sollte das System auch weiterhin transportabel - also nicht zu schwer - und veränderbar bleiben, beispielsweise Kameras für Messreihen in ihrer Lage verändert werden. Der Aufbau aus genormten Bauteilen bietet sich an, um hier ggf. den Nachbau einfach ermöglichen zu können.

Als mögliche Materialien kamen Holz, Stahl und Aluminum in Frage. Aufgrund der einfachen Bearbeitung und der genormten Profile, wurde sich für Aluminiumprofile entschieden. Diese gibt es in verschiedenen Ausführungen mit Nuten an den Seitenflächen, so dass eine einfache Montage, aber auch eine Demontage zu Transportzwecken, möglich wird. Außerdem sind diese sehr stabil bei leichtem Gewicht.

Durch eine Konstruktion mit Eckwürfeln sowie dem Einbau von dreieckigen Strukturen und Platten, die Scheibenwirkung haben, wurde die Stabilität der Verbindungen erhöht.

\section{Beleuchtung}
Um möglichst gute Bilder zu erzeugen, sollte das Objekt gut ausgeleuchtet sein. Eine dunkle Umgebung verlängert die Belichtungszeit, wodurch die Gefahr von unscharfen Aufnahmen steigt und dunklere Bereiche (ungleichmäßige Ausleuchtung) verursacht Rauschen in diesen Bildbereichen. Daher soll das System eine gleichmäßige Ausleuchtung ermöglichen. Problematisch ist hierbei, dass die Kameras ggf. auch die Lichtquellen mit im Bildbereich haben können, wodurch Linsenreflexionen oder Ausbrennen der Bildbereiche möglich sind.

Es wurde sich zur Beleuchtung für einzeln steuerbare LED-Lichtstreifen entschieden. Diese können einfach an den Aluprofilen montiert werden und ermöglichen es, einzelne Bereiche und so ggf. blendende Bereiche abzuschalten.

\section{Stromversorgung}
Die Raspberry Pis werden mit 5~Volt betrieben. Der Raspberry Pi Zero mit Kamera hatte dabei in Messungen einen maximalen Stromverbrauch von 270~mA aufgezeigt, der Raspberry Pi 4 kann bis zu 1,5~A unter Last verbrauchen. Hieraus ergibt sich ein Gesamtstromverbrauch von maximal rund 8~Ampere. Für den Raspbery Pi 4 wurde ein eigenes Netzteil eingeplant und für die Zero W ein gemeinsames 35 Watt-Netzteil. Versuche zeigten jedoch, dass der Stromverbrauch kurzfristig höher ausfallen konnte, so dass die Zero W, die am meisten von Spannungsabfällen betroffen sind zum Absturz gebracht wurden, wenn alle Kameras gleichzeitig auslösten. Nach dem die Last auf sicherheitshalber auf zwei weitere Netzteile verteilt wurde, lief das System zuverlässig.

Als Kabelmaterial wurde Klingeldraht mit $0,75~\text{mm}^2$ verwendet. Der relativ hohe Kabelquerschnitt soll für einen geringen Spannungsabfall sorgen. Durch die Verwendung von mehreren Netzteilen ist dieser jedoch nun nicht mehr notwendig. Hier würde sich nun ein geringerer Querschnitt anbieten, auch um eine einfachere Verbindung zu den Raspberry Pi Zero W zu ermöglichen. Diese wurden auf Seiten der Zero W verlötet und in den Verteilerdosen mit Federkraftklemmen verbunden.

Die Stromversorgung der Beleuchtung erfolgt über ein 12-V-Netzteil mit $3,5~\text{A}$ Ausgangsleistung. Auch hier wurde Klingeldraht zur Verteilung zwischen den einzelnen Holmen genutzt.

Beim WLAN-Router war ein entsprechendes USB-Netzteil mitgeliefert.

\section{Kommunikation und Datenübertragung}
Die Kommunikation zwischen den Raspberry Pis erfolgt über WLAN. Vorteil dieser Lösung ist, dass hier keine weiteren Leitungen außer der Stromversorgung zu den einzelnen Raspberry Pi Zero W benötigt wird und es auch möglich wäre, die gleiche Hard- und Software auch für ein größeres System ohne Änderungen zu nutzen. Nachteilig ist die Verbindungsgeschwindigkeit, gerade im Hinblick auf die Synchronisierung der Kameras. Diese Problematik soll aber durch entsprechende Programmierung der Software möglichst klein gehalten werden.

Als weitere Datenleitungen wird eine Steuerleitung für die LED-Streifen benötigt. Über diese erfolgt die Steuerung der einzelnen LED-Gruppen. Auch hier wurde wieder Klingeldraht verwendet.


\end{document}