\documentclass[./00PhotoBox.tex]{subfiles}
\graphicspath{{\subfix{./img/}}}
\begin{document}


\chapter{Systemkalibrierung}
Der Maßstab eines rein photogrammetrisch bestimmten 3D-Modelles ist nicht bekannt. Daher werden mindestens Referenzen in Form einer bekannten Länge benötigt um den Maßstab zu bestimmen. Neben dieser mindestens notwendigen Referenz sind weitere Kalibrierungen der Kamera möglich, um die Kameraparameter zu bestimmen. Da die Kameras jedoch alleine schon durch ihren Autofokus nicht stabil in ihrer inneren Orientierung sind, ist eine Kalibrierung der Kamera nur bedingt möglich und bietet nur einen Näherungswert.

\begin{itemize}
    \item Kameramodellierung
    \item Kamerakalibrierung
    \item Kameraausrichtung
    \item Farbkalibrierung
\end{itemize}

\biblio
\end{document}