\documentclass[./00PhotoBox.tex]{subfiles}
\graphicspath{{\subfix{./img/}}}

\begin{document}

\chapter{Kurzbedienungsanleitung}

Für detaillierte Erklärungen und Problemlösungen lesen Sie bitte die Bedienungsanleitung. Hintergrundinformationen sind der Thesis zu entnehmen.


\begin{enumerate}
    \item Stromversorgung herstellen, ggf. Steckdosenleiste einschalten
    \item Warten bis alle LEDs grün aufleuchten, evtl. mittels kurzem Druck auf Status-Taste überprüfen
    \item Gewünschte Ausgabe anschließen
          \begin{itemize}
              \item Computer per WLAN verbinden
              \item USB-Stick an Raspberry Pi 4 anschließen
          \end{itemize}
    \item Aufnahmeknopf drücken
    \item Warten bis alle LEDs grün aufleuchten, Bilder werden übertragen
    \item Weiterverarbeitung auf Computer
          \begin{itemize}
              \item au\-to\-ma\-tisch, wenn Computer per WLAN verbunden
              \item manuell, wenn USB-Stick verwendet
          \end{itemize}
    \item System herunterfahren
          \begin{itemize}
              \item Langen Druck auf roten Taster
              \item Warten bis alle LEDs erlöschen
              \item Stromversorgung trennen
          \end{itemize}
\end{enumerate}

\section{Agisoft Metashape}
\begin{enumerate}
    \item Metashape starten
    \item Projekt öffnen
    \item Haken bei allen Passpunkten und Kameras setzen
    \item \textit{Ablauf / Fotos ausrichten...}
    \item Bereich des Modells festlegen
    \item \textit{Ablauf / Punktwolke erzeugen...}
    \item ggf. \textit{Ablauf / Mesh erzeugen...}
    \item ggf. \textit{Ablauf / Textur erzeugen...}
    \item \textit{Datei / Exportieren / Modell exportieren...} oder \textit{Punktwolke exportieren...}
\end{enumerate}

\biblio
\end{document}