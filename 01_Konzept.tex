\documentclass[./00_PhotoBox.tex]{subfiles}
\graphicspath{{\subfix{./img/}}}
\begin{document}


\chapter{Konzept}
In Museen besteht vielfach der Wunsch, ihre Exponate zu digitalisieren. Entsprechende Handreichungen des Deutschen Museumsbundes legen auch die Digitalisierung als 3D-Modelle nahe, verweisen aber auf Aufwand und Format-Probleme \citep[S. 43]{handreichung_digital}.
Auch bei Ausgrabungen aber auch in anderen Bereichen besteht der Bedarf dreidimensionale Modelle einfach und kostengünstig zu erfassen.

Diese Möglichkeit soll das im Rahmen dieser Arbeit entwickelte Kamerasystem bieten. Es soll mittels Photogrammetrie ermöglichen, mit geringen personellen Aufwand kleine Objekte bis etwa 40~cm Durchmesser zu erfassen. Die Bedienung soll dabei laiensicher und mit nur kurzer Einarbeitungszeit möglich sein, dass System also die meisten Schritte selbstständig durchführen. Auch der Nachbau soll mit etwas handwerklichen Geschick möglich sein. Um Lizenzkosten zu sparen, soll die Möglichkeit auf OpenSource-Software zu setzen geprüft werden.

Neben der eigentlichen Entwicklung eines funktionsfähigen Systemes soll abschließend die Anzahl der Kameras und die Nutzung eines Drehtellers evaluiert werden, um hiermit gegebenenfalls die Hardwarekosten zu senken oder die Auflösung und Genauigkeit zu steigern.

\end{document}